\chapter{Problemas a resolver}
Ejercicios propuestos al lector obtenidos del libro: \textit{Trascendentes tempranas, de James Stewart.}
\begin{itemize}
 \item Cuando la sangre fluye por un vaso sanguíneo, el flujo F (el
volumen de sangre por unidad de tiempo que corre por un
punto dado) es proporcional a la cuarta potencia del radio R de
ese vaso:
$$F=kR^4.$$
Una arteria parcialmente
obstruida puede expandirse por medio de una operación
llamada angioplastia, en la cual un catéter provisto de un
globo en la punta se infla dentro del vaso a fin de ensancharlo
y restituir el flujo sanguíneo normal.
 Demuestre que el cambio relativo en F es alrededor de cuatro
veces el cambio relativo en R. ¿Cómo afectará un aumento de
5 \% en el radio al flujo de sangre? (Stewart, 2016, p. 256).
\item
Use la ley de Poiseuille para calcular la razón del flujo sanguíneo
en una pequeña arteria humana donde puede tomarse
 $\eta=0.027,\ R=0.008 cm,\ l=2cm\ y\ P=4000\ \tfrac{dinas}{cm^2}$\\ (Stewart, 2016, p. 597).
 \item
 La presión sanguínea alta resulta de la obstrucción de las
arterias. Para mantener un flujo normal, el corazón tiene que
bombear más fuerte, de modo que se incrementa la presión
arterial. Use la ley de Poiseuille para demostrar que si $R_0$ y $P_0$
son valores normales del radio y la presión en una arteria, y
los valores obstruidos son R y P, respectivamente, entonces
para que el flujo permanezca constante, P y R se relacionan
mediante la ecuación

\begin{align*}
    \frac{P}{P_0}=\left(\frac{R_0}{R}\right)^4
\end{align*}
Deduzca que si el radio de una arteria se reduce a tres cuartos
de su valor anterior, entonces la presión es más que el triple (Stewart, 2016, p. 567).

Como observación al ejercicio anterior, opino que sería complementario que el lector experimente con la expresión matemática del ejercicio, es decir, que se cuestione sobre qué pasaría si el radio disminuye 2/5, en vez de 3/4 del valor anterior, además de que investigue por su propia cuenta casos clínicos donde cierto valor de reducción del radio de la arteria (relacionado a una patología) resulta mortal en algún paciente, y cómo es que se puede evitar.
\end{itemize} 